\documentclass[11pt]{article}

\usepackage{listings,color,amsmath,xfrac,amssymb}
\usepackage[hidelinks]{hyperref}
\definecolor{comments}{RGB}{81,81,81}
\definecolor{keywords}{RGB}{255,0,90}

% lstlisting
\lstset{
    language=C,
    basicstyle=\footnotesize\ttfamily,
    keywordstyle=\color{keywords},
    showspaces=false,
    showstringspaces=false,
    commentstyle=\color{blue}\emph
}

\usepackage[
    type={CC},
    modifier={by-nc-nd},
    version={4.0},
]{doclicense} 


\topmargin -.5in
\textheight 9in
\oddsidemargin -.25in
\evensidemargin -.25in
\textwidth 7in

\newcommand{\courseurl}[0]{https://www.cct.lsu.edu/\string~pdiehl/teaching/2021/4997/}
\newcommand{\coursetimeline}[0]{https://www.cct.lsu.edu/~pdiehl/teaching/2021/4997/timeline.pdf}
\newcommand{\coursesyllabus}[0]{https://www.cct.lsu.edu/~pdiehl/teaching/2021/4997/syllabus.pdf}
\newcommand{\coursename}[0]{Math 4997-3}
\newcommand{\coursemailinglist}[0]{https://mail.cct.lsu.edu/mailman/listinfo/par4997}
\newcommand{\coursedate}{2021-08-24}








\ifxetex
\usepackage{fontspec}
\setmainfont{Raleway}
\fi

\begin{document}

% ========== Edit your name here
%\author{Your Name}
\title{\coursename~Quiz 2: Due by Tuesday, September 10}
\date{}
\maketitle

\medskip

% ========== Begin answering questions here

\section*{Exercises}

\begin{enumerate}
\item Programming on paper (2 credits): \\
Write a program that computes the median of the elements in a vector.

\item Interpreting programs (2 credits): \\
What does this program do?
\lstinputlisting{code/exercise2-1.cpp}


\end{enumerate}

\section*{Programming exercises}

\begin{enumerate}

\item Monte Carlo method: (4 credits)\\
In Lecture 2, we discussed the Monte Carlo Method to estimate the value of $\pi$ by
\begin{enumerate}
\item Read $n$ from the terminal
\item Generate random coordinates $(x,y)\in \mathbb{R}^2$
\item Check if $x^2+y^2 \leq 1$
\begin{itemize}
\item Update $N_c$ if $\leq 1$
\end{itemize}
\item Increment $n$ the interval  
\item If $n$ < $n_{\text{total}}$ go to 1
\item Calculate $\pi \approx 4 \sfrac{N_c}{n}$
\item Print result
\end{enumerate}

\item Measuring time: (2 credits)\\
To measure the computation time, one can use the timers \lstinline|std::chrono::high_resolution_clock| of the \lstinline|#include <chrono>| header\footnote{\url{https://en.cppreference.com/w/cpp/chrono/high_resolution_clock}}.
\begin{lstlisting}
// Get starting timepoint 
auto start = std::chrono::high_resolution_clock::now(); 
// Do work
// Stop timer
auto stop = high_resolution_clock::now();
// Get the duration
auto duration = duration_cast<microseconds>(stop - start); 
// Print the execution time  
cout << "Time taken by function: "
     << duration.count() << " microseconds" << endl; 
\end{lstlisting} 
Write a program that fills a vector and a list with $n$ elements and measure the execution time of both and print them to the terminal. 

% ========== Continue adding items as needed

\end{enumerate}

\doclicenseThis 

\end{document}
\grid
\grid
