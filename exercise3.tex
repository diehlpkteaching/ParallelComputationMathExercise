\documentclass[11pt]{article}

\usepackage{listings,color,amsmath,xfrac,amssymb}
\usepackage[hidelinks]{hyperref}
\definecolor{comments}{RGB}{81,81,81}
\definecolor{keywords}{RGB}{255,0,90}

% lstlisting
\lstset{
    language=C,
    basicstyle=\footnotesize\ttfamily,
    keywordstyle=\color{keywords},
    showspaces=false,
    showstringspaces=false,
    commentstyle=\color{blue}\emph
}

\usepackage[
    type={CC},
    modifier={by-nc-nd},
    version={4.0},
]{doclicense} 


\topmargin -.5in
\textheight 9in
\oddsidemargin -.25in
\evensidemargin -.25in
\textwidth 7in

\newcommand{\courseurl}[0]{https://www.cct.lsu.edu/\string~pdiehl/teaching/2021/4997/}
\newcommand{\coursetimeline}[0]{https://www.cct.lsu.edu/~pdiehl/teaching/2021/4997/timeline.pdf}
\newcommand{\coursesyllabus}[0]{https://www.cct.lsu.edu/~pdiehl/teaching/2021/4997/syllabus.pdf}
\newcommand{\coursename}[0]{Math 4997-3}
\newcommand{\coursemailinglist}[0]{https://mail.cct.lsu.edu/mailman/listinfo/par4997}
\newcommand{\coursedate}{2021-08-24}








\ifxetex
\usepackage{fontspec}
\setmainfont{Raleway}
\fi

\begin{document}

% ========== Edit your name here
%\author{Your Name}
\title{\coursename~Quiz 3: Due by Tuesday, September 10}
\date{}
\maketitle

\medskip

% ========== Begin answering questions here

\section*{Exercises}

\begin{enumerate}
\item Programming on paper (2 credits): \\
Write a struct for a complex number and overload the +,-, and * operator.

\item Interpreting programs (2 credits): \\
What does this program do? Please write down the value of $n$ at each occurrence of the \lstinline|std::cout| statement.
\lstinputlisting{code/exercise3-2.cpp}


\end{enumerate}

\section*{Programming exercise}

\begin{enumerate}

\item N-body problem: (6 credits)\\
In this exercise, we will implement the N-Body simulating using a direct sum to compute the forces and the Euler Method to update the positions.
\begin{itemize}
\item Datastrucutre:
\begin{enumerate}
\item Write a generic struct for a vector
\item Add a function to compute the vector's norm
\item Add a constructor which initializes the vector to zero
\item Overload the operators +,-, and == for a second vector and the operator * for multplication with a scalar
\end{enumerate}
\item Class for the $N$-body simulation
\begin{enumerate}
\item Write a function to compute the force $\mathbf{F}_i=\sum\limits_{i=0,i\neq j}^n \mathbf{F}_{ij}= \sum\limits_{i=0,i\neq j}^n g_c m_j \frac{\mathbf{r}_j-\mathbf{r}_i}{\vert \mathbf{r}_j - \mathbf{r}_i\vert^3}$ for each body using a direct sum
\item Write a function to compute the velocity $\mathbf{v}_i(t_k)=\mathbf{v}_i(t_{k-1})+\Delta t \frac{\mathbf{F}_i}{m_i}$ for each body
\item Write a function to update the new positions $\mathbf{r}_i(t_{k+1})=\mathbf{r}_i(t_k)+\mathbf{v}_i(t_k)\Delta t$ of a each body
\end{enumerate}
\end{itemize}



Note that you will get an invitation via Github classroom and you should use this repository to submit your solution. In addition, you can find the templates here\footnote{\url{https://github.com/diehlpkteaching/N-Body}}.


% ========== Continue adding items as needed

\end{enumerate}

\doclicenseThis 

\end{document}
\grid
\grid
