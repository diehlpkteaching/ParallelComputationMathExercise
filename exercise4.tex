\documentclass[11pt]{article}

\usepackage{listings,color,amsmath,xfrac,amssymb}
\usepackage[hidelinks]{hyperref}
\definecolor{comments}{RGB}{81,81,81}
\definecolor{keywords}{RGB}{255,0,90}

% lstlisting
\lstset{
    language=C,
    basicstyle=\footnotesize\ttfamily,
    keywordstyle=\color{keywords},
    showspaces=false,
    showstringspaces=false,
    commentstyle=\color{blue}\emph
}

\usepackage[
    type={CC},
    modifier={by-nc-nd},
    version={4.0},
]{doclicense} 


\topmargin -.5in
\textheight 9in
\oddsidemargin -.25in
\evensidemargin -.25in
\textwidth 7in

\newcommand{\courseurl}[0]{https://www.cct.lsu.edu/\string~pdiehl/teaching/2021/4997/}
\newcommand{\coursetimeline}[0]{https://www.cct.lsu.edu/~pdiehl/teaching/2021/4997/timeline.pdf}
\newcommand{\coursesyllabus}[0]{https://www.cct.lsu.edu/~pdiehl/teaching/2021/4997/syllabus.pdf}
\newcommand{\coursename}[0]{Math 4997-3}
\newcommand{\coursemailinglist}[0]{https://mail.cct.lsu.edu/mailman/listinfo/par4997}
\newcommand{\coursedate}{2021-08-24}








\ifxetex
\usepackage{fontspec}
\setmainfont{Raleway}
\fi

\begin{document}

% ========== Edit your name here
%\author{Your Name}
\title{\coursename~Quiz 4: Due by Tuesday, September 24}
\date{}
\maketitle

\medskip

% ========== Begin answering questions here

\section*{Exercises}

\begin{enumerate}
\item Programming on paper (2 credits): \\
Write a program that outputs the Fibonacci sequence up to a number $N$. The Fibonacci sequence is given as
\begin{align*}
F(0) &= 0 \text{ and } F(1) = 1  \\
F(n) &= F(n-1) + F(n-2) \; \forall n > 1\text{.}
\end{align*} 

\item Definitions (2 credits): \\
Explain the following terms in your own words:
\begin{itemize}
\item Race condition and data race (1 credit)
\item Deadlock (1 credit)
\end{itemize}


\end{enumerate}

\section*{Programming exercise}

\begin{enumerate}

\item CMake: (2 credits)\\
Use any of your previous exercises and write a CMake file to compile this exercise.
\item Numerical integration (4 credits)\\
The trapezoidal rule can be used to approximate the definite integral 
\begin{align*}
\int\limits_a^b f(x) dx \approx \frac{h}{2} \sum\limits_{k=1}^N (f(x_{k-1}) + f(x_k))
\end{align*}
assuming a uniform grid in the interval $[a,b]$ with the grid size $h=\frac{b-a}{N}$.
\begin{enumerate}
\item Write a C++ program which approximates the integral for $f(x)=x^2$ on $[0,2]$ for any given $N$. (2 credits)
\item Use \lstinline|std::future| and \lstinline|std::async| compute the solution asynchronously. (2 credit)
\end{enumerate}
Validate both of your implementations against the solution 
\begin{align*}
\int\limits_0^2 x^2 = \left\vert \frac{x^3}{3}\right\vert_0^2 = \frac{2^3}{3} - \frac{0^2}{3} = \frac{8}{3}\text{.}
\end{align*}
% ========== Continue adding items as needed

\end{enumerate}

\doclicenseThis 

\end{document}
\grid
\grid
