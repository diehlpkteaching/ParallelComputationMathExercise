\documentclass[11pt]{article}

\usepackage{listings,color,amsmath,xfrac,amssymb}
\usepackage[hidelinks]{hyperref}
\definecolor{comments}{RGB}{81,81,81}
\definecolor{keywords}{RGB}{255,0,90}

% lstlisting
\lstset{
    language=C,
    basicstyle=\footnotesize\ttfamily,
    keywordstyle=\color{keywords},
    showspaces=false,
    showstringspaces=false,
    commentstyle=\color{blue}\emph
}

\usepackage[
    type={CC},
    modifier={by-nc-nd},
    version={4.0},
]{doclicense} 


\topmargin -.5in
\textheight 9in
\oddsidemargin -.25in
\evensidemargin -.25in
\textwidth 7in

\input{variables.tex}

\ifxetex
\usepackage{fontspec}
\setmainfont{Raleway}
\fi

\begin{document}

% ========== Edit your name here
%\author{Your Name}
\title{\coursename~Quiz 5: Due by Tuesday, October 1}
\date{}
\maketitle

\medskip

% ========== Begin answering questions here

\section*{Exercises}

\begin{enumerate}
\item Programming on paper (2 credits): \\
Write a program that computes
\begin{align*}
A = B + C
\end{align*}
where $A,B,C$ are \lstinline|std::vector| in parallel.

\item Definitions (2 credits): \\
Explain the following terms in your own words:
\begin{itemize}
\item Asynchronous vs synchronous programming
\item Explain what a \lstinline|std::future| is and how to utilized it for parallism in your application.
\end{itemize}


\end{enumerate}

\section*{Programming exercise}

\begin{enumerate}

\item Parallel Monte-Carlo methods: (2 credits)\\
Use your solution of the $N$-body solution and add parallelism to your implementation using \\
\lstinline|std::execution::par|, \lstinline|std::async|, and \lstinline|std::future|. Try to replace as many as possible of the \lstinline|lop| with parallel lops. Try to launch some of the functions asynchronously and synchronize them using the future objects. 

\item Parallel $N$-body simulation (4 credits)\\
Use your solution of the $N$-body solution and add parallelism to your implementation using \\
\lstinline|std::execution::par|, \lstinline|std::async|, and \lstinline|std::future|. Try to replace as many as possible of the \lstinline|lop| with parallel lops. Try to launch some of the functions asynchronously and synchronize them using the future objects. 
% ========== Continue adding items as needed

\end{enumerate}
Please contact me, if you need the solutions of these programming exercises.
\doclicenseThis 

\end{document}
\grid
\grid
