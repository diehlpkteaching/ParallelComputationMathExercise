    \documentclass[11pt]{article}

\usepackage{listings,color,amsmath,xfrac,amssymb}
\usepackage[hidelinks]{hyperref}
\definecolor{comments}{RGB}{81,81,81}
\definecolor{keywords}{RGB}{255,0,90}

% lstlisting
\lstset{
    language=C,
    basicstyle=\footnotesize\ttfamily,
    keywordstyle=\color{keywords},
    showspaces=false,
    showstringspaces=false,
    commentstyle=\color{blue}\emph
}

\usepackage[
    type={CC},
    modifier={by-nc-nd},
    version={4.0},
]{doclicense} 


\topmargin -.5in
\textheight 9in
\oddsidemargin -.25in
\evensidemargin -.25in
\textwidth 7in

\newcommand{\courseurl}[0]{https://www.cct.lsu.edu/\string~pdiehl/teaching/2021/4997/}
\newcommand{\coursetimeline}[0]{https://www.cct.lsu.edu/~pdiehl/teaching/2021/4997/timeline.pdf}
\newcommand{\coursesyllabus}[0]{https://www.cct.lsu.edu/~pdiehl/teaching/2021/4997/syllabus.pdf}
\newcommand{\coursename}[0]{Math 4997-3}
\newcommand{\coursemailinglist}[0]{https://mail.cct.lsu.edu/mailman/listinfo/par4997}
\newcommand{\coursedate}{2021-08-24}








\ifxetex
\usepackage{fontspec}
\setmainfont{Raleway}
\fi

\begin{document}

% ========== Edit your name here
%\author{Your Name}
\title{\coursename~Quiz 6: Due by Tuesday, October 8}
\date{}
\maketitle

\medskip

% ========== Begin answering questions here

\section*{Exercises}

\begin{enumerate}
\item Programming on paper (2 credits): \\
Write a program that squares all elements in a \lstinline|std::vector<double>| and compute the sum of all elements using \lstinline|std::for_each| and \lstinline|std::execution::par|.


\item Definitions (2 credits): \\
What does this program do?
\begin{lstlisting}
#include <iostream>
#include <vector>
#include <numeric>
#include <future>

using namespace std;

int func1(vector<int> values){
    
    
    return accumulate(values.begin(),values.end(),0);
    
    
}

int main()
{
    std::vector<int> values = {1,2,3,4,5,6,7,8,9,10};
    
    auto f1 = std::async(func1,values);
    
    auto f2 = std::async([](const vector<int> values )
          {return std::inner_product(values.begin(), values.end(), values.begin(), 0);}
          ,values);
    
    
    cout << f1.get() + f2.get() << std::endl;

    return 0;
}
\end{lstlisting}


\end{enumerate}

\section*{Programming exercise}

\begin{enumerate}

\item Communication matrix: (2 credits)\\
Use the following matrix as the network of people
\begin{center}
$
\mathbf{M} = \left[\begin{matrix}
1 & 1 & 1 & 1 & 1 \\
1 & 0 & 0 & 1 & 0 \\
0 & 1 & 1 & 0 & 0 \\
1 & 1 & 1 & 1 & 1 \\
0 & 0 & 1 & 1 & 0
\end{matrix}\right]
$
\end{center}
and compute $\mathbf{M}^4$, where $\mathbf{M}^4= \mathbf{M}*\mathbf{M}*\mathbf{M}*\mathbf{M}$. Ask the user to provide you with a value of $n$ and print the resulting matrix to the terminal.

\item Conjugate gradient method (4 credits)\\
To solve a equation system $\mathbf{A}\mathbf{x}=\mathbf{b}$, we can use the conjugate gradient methods (CG) by using following algorithm
\begin{enumerate}
\item $\mathbf{r_0} = \mathbf{b} - \mathbf{A} \mathbf{x}_0$
\item If $\mathbf{r}_0< \epsilon$ return  $\mathbf{x}_0$
\item $\mathbf{p}_0=\mathbf{r}_0$
\item $k=0$
\item $\alpha_k = \frac{\mathbf{r}_k^T\mathbf{r}_k}{\mathbf{p}_k^T\mathbf{A}\mathbf{p}_k}$
\item $ \mathbf{x}_{k+1} = \mathbf{x}_k + \alpha_k \mathbf{p}_k$
\item $ \mathbf{r}_{k+1} = \mathbf{r}_k - \alpha_k \mathbf{p}_k$
\item If $\mathbf{r}_{k+1}< \epsilon$ exit loop
\item $\beta_k = \frac{\mathbf{r}_{k+1}^T\mathbf{r}_{k+1}}{\mathbf{r}_k^T\mathbf{r}_k}$
\item $\mathbf{p}_{k+1}=\mathbf{r}_{k+1} + \beta_k \mathbf{p}_k$
\item $k=k+1$
\item go to (e)
return $\mathbf{x}_{k+1}$
\end{enumerate}
Implement the conjugate gradient algorithm using the Blaze library\footnote{\url{https://bitbucket.org/blaze-lib/blaze/wiki/Home}}. Note that the Blaze library is installed on the course's server and I recommend to use the server or Jupyter notebooks for this exercise.

Note that $\mathbf{x}_0$ can be chosen, if we know some assumption of the solution or set to zero. The symbol $\mathbf{v}^T$ denotes the transpose of the vector $\mathbf{v}$. Please use this code testing your implementation
\begin{lstlisting}
 for(int i=0; i<N; ++i) {
        A(i,i) = 2.0;
        b[i] = 1.0*(1+i);
        x1[i] += x1[i-1];
    }
\end{lstlisting}



% ========== Continue adding items as needed

\end{enumerate}
\doclicenseThis 

\end{document}
\grid
\grid
