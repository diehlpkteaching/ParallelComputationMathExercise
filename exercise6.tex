    \documentclass[11pt]{article}

\usepackage{listings,color,amsmath,xfrac,amssymb}
\usepackage[hidelinks]{hyperref}
\definecolor{comments}{RGB}{81,81,81}
\definecolor{keywords}{RGB}{255,0,90}

% lstlisting
\lstset{
    language=C,
    basicstyle=\footnotesize\ttfamily,
    keywordstyle=\color{keywords},
    showspaces=false,
    showstringspaces=false,
    commentstyle=\color{blue}\emph
}

\usepackage[
    type={CC},
    modifier={by-nc-nd},
    version={4.0},
]{doclicense} 


\topmargin -.5in
\textheight 9in
\oddsidemargin -.25in
\evensidemargin -.25in
\textwidth 7in

\input{variables.tex}

\ifxetex
\usepackage{fontspec}
\setmainfont{Raleway}
\fi

\begin{document}

% ========== Edit your name here
%\author{Your Name}
\title{\coursename~Quiz 5: Due by Tuesday, October 1}
\date{}
\maketitle

\medskip

% ========== Begin answering questions here

\section*{Exercises}

\begin{enumerate}
\item Programming on paper (2 credits): \\
Write a program that computes
\begin{align*}
A = B + C
\end{align*}
where $A,B,C$ are \lstinline|std::vector| in parallel.

\item Definitions (2 credits): \\
Explain the following terms in your own words:
\begin{itemize}
\item Asynchronous vs synchronous programming
\item Explain what a \lstinline|std::future| is and how to utilized it for parallism in your application.
\end{itemize}


\end{enumerate}

\section*{Programming exercise}

\begin{enumerate}

\item Communication matrix: (2 credits)\\
Use the following matrix as the network of people
\begin{center}
$
M = \left[\begin{matrix}
1 & 1 & 1 & 1 & 1 \\
1 & 0 & 0 & 1 & 0 \\
0 & 1 & 1 & 0 & 0 \\
1 & 1 & 1 & 1 & 1 \\
0 & 0 & 1 & 1 & 0
\end{matrix}\right]
$
\end{center}
and compute $M^n$, where $m^2_{3,2}=a_{3,1}a_{1,2}+a_{3,2}a_{2,2}+a_{3,4}a_{4,2}$. Ask the user to provide you with a value of $n$ and print the resulting matrix to the terminal.

\item Conjugate gradient method (4 credits)\\
To solve a equation system $\mathbf{A}\mathbf{x}=\mathbf{b}$, we can use the conjugate gradient methods (CG) by using following algorithm
\begin{enumerate}
\item $\mathbf{r_0} = \mathbf{b} - \mathbf{A} \mathbf{x}_0$
\item If $\mathbf{r}_0< \epsilon$ return  $\mathbf{x}_0$
\item $\mathbf{p}_0=\mathbf{r}_0$
\item $k=0$
\item $\alpha_k = \frac{\mathbf{r}_k^T\mathbf{r}_k}{\mathbf{p}_0^T\mathbf{A}\mathbf{p}_k}$
\item $ \mathbf{x}_{k+1} = \mathbf{x}_k + \alpha_k \mathbf{p}_k$
\item $ \mathbf{r}_{k+1} = \mathbf{r}_k + \alpha_k \mathbf{p}_k$
\item If $\mathbf{r}_{k+1}< \epsilon$ exit loop
\item $\beta_k = \frac{\mathbf{r}_{k+1}^T\mathbf{r}_{k+1}}{\mathbf{r}_k^T\mathbf{r}_k}$
\item $\mathbf{p}_{k+1}=\mathbf{r}_{k+1} + \beta_k \mathbf{p}_k$
\item $k=k+1$
\item go to (e)
return $\mathbf{x}_{k+1}$
\end{enumerate}
Implement the conjugate gradient algorithm using the Blaze library.

Note that $\mathbf{x}_0$ can be chosen, if we know some assumption of the solution or set to zero. The symbol $mathbf{v}^T$ denotes the transpose of the vector $\mathbf{v}$.
% ========== Continue adding items as needed

\end{enumerate}
\doclicenseThis 

\end{document}
\grid
\grid
