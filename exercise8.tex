\documentclass[11pt]{article}

\usepackage{listings,color,amsmath,amssymb}
\usepackage[hidelinks]{hyperref}

\definecolor{comments}{RGB}{81,81,81}
\definecolor{keywords}{RGB}{255,0,90}

% lstlisting
\lstset{
    language=C,
    basicstyle=\footnotesize\ttfamily,
    keywordstyle=\color{keywords},
    showspaces=false,
    showstringspaces=false,
    commentstyle=\color{blue}\emph
}


\usepackage[
    type={CC},
    modifier={by-nc-nd},
    version={4.0},
]{doclicense} 


\topmargin -.5in
\textheight 9in
\oddsidemargin -.25in
\evensidemargin -.25in
\textwidth 7in

\newcommand{\courseurl}[0]{https://www.cct.lsu.edu/\string~pdiehl/teaching/2021/4997/}
\newcommand{\coursetimeline}[0]{https://www.cct.lsu.edu/~pdiehl/teaching/2021/4997/timeline.pdf}
\newcommand{\coursesyllabus}[0]{https://www.cct.lsu.edu/~pdiehl/teaching/2021/4997/syllabus.pdf}
\newcommand{\coursename}[0]{Math 4997-3}
\newcommand{\coursemailinglist}[0]{https://mail.cct.lsu.edu/mailman/listinfo/par4997}
\newcommand{\coursedate}{2021-08-24}








\usepackage{ifxetex}

\ifxetex
\usepackage{fontspec}
\setmainfont{Raleway}
\fi

\usepackage[calc,datesep=/]{datetime2}
\newcount\daycount
\newcommand{\DueDate}[2]{%
  \DTMsavedate{ShootDate}{#1}%
  \DTMsaveddateoffsettojulianday{ShootDate}{#2}\daycount
  \DTMsavejulianday{ShootDate}{\number\daycount}%
  \DTMusedate{ShootDate}%
}

\usepackage{xfrac}

\begin{document}

% ========== Edit your name here
%\author{Your Name}
\title{\coursename~Quiz 8: Due by \DueDate{\coursedate}{65}}
\date{}
\maketitle

\medskip

% ========== Begin answering questions here

\section*{Exercises}

\begin{enumerate}
\item Enhanced futures (4 credits): \\
HPX implements following two additional \lstinline|future| variants
\begin{itemize}
\item \lstinline|hpx::shared_future|
\item \lstinline|hpx::make_ready_future|
\end{itemize}
Please explain why HPX provides these additional future variants. Try to make a small example, where these are needed.

\section*{Programming exercise}

\begin{itemize}
\item Using dataflow (4 credits): \\
Replace the \lstinline|hpx::when_all(futures).then()| by hpx::dataflow. Note that the function \lstinline|square| is defined in the next source code listing.
\begin{lstlisting}
run_hpx([](){

std::vector<hpx::lcos::future<int>> futures;
futures.push_back(hpx::async(square,10));
futures.push_back(hpx::async(square,100));

hpx::when_all(futures).then([](auto&& f){
    std::vector<hpx::lcos::future<int>> futures = f.get();
    int result = 0;
    for(size_t i = 0; i < futures.size();i++)
        result += futures[i].get();
    std::cout << result << std::endl;
});

});
\end{lstlisting}

\item Unwrapping futures: (2 credits)\\
In the code below are two comments and two lines of code are missing there. Please add the two missing lines of code.
\begin{lstlisting}
// Evaluate the result and print the result
void sum(int first, int second){
	std::cout << first + second << std::endl;
}
// Compute the square 
int square(int a)
{
return a*a ;
}

int main(){
	std::vector<hpx::lcos::future<int>> futures;
	futures.push_back(hpx::async(square,10));
	futures.push_back(hpx::async(square,100));
	
	// Unwrapp the function sum
	
	// Use dataflow to execute the function asynchronously.
	
		
	return 0;
}
\end{lstlisting} 
\end{itemize}

% ========== Continue adding items as needed

\end{enumerate}
\doclicenseThis 

\end{document}
\grid
\grid
