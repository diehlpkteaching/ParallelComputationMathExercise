\documentclass[11pt]{article}

\usepackage{listings,color,amsmath,xfrac,amssymb}
\usepackage[hidelinks]{hyperref}
\definecolor{comments}{RGB}{81,81,81}
\definecolor{keywords}{RGB}{255,0,90}

% lstlisting
\lstset{
    language=C,
    basicstyle=\footnotesize\ttfamily,
    keywordstyle=\color{keywords},
    showspaces=false,
    showstringspaces=false,
    commentstyle=\color{blue}\emph
}

\usepackage[
    type={CC},
    modifier={by-nc-nd},
    version={4.0},
]{doclicense} 


\topmargin -.5in
\textheight 9in
\oddsidemargin -.25in
\evensidemargin -.25in
\textwidth 7in

\newcommand{\courseurl}[0]{https://www.cct.lsu.edu/\string~pdiehl/teaching/2021/4997/}
\newcommand{\coursetimeline}[0]{https://www.cct.lsu.edu/~pdiehl/teaching/2021/4997/timeline.pdf}
\newcommand{\coursesyllabus}[0]{https://www.cct.lsu.edu/~pdiehl/teaching/2021/4997/syllabus.pdf}
\newcommand{\coursename}[0]{Math 4997-3}
\newcommand{\coursemailinglist}[0]{https://mail.cct.lsu.edu/mailman/listinfo/par4997}
\newcommand{\coursedate}{2021-08-24}








\ifxetex
\usepackage{fontspec}
\setmainfont{Raleway}
\fi

\begin{document}

% ========== Edit your name here
%\author{Your Name}
\title{\coursename~Quiz 8: Due by Tuesday, November 05}
\date{}
\maketitle

\medskip

% ========== Begin answering questions here

\section*{Exercises}

\begin{enumerate}
\item Futurization of the partition-based 1D heat equation (10 credits): \\
In this exercise you should understand the new version of the code of the 1D heat equation using futurization. You can use the example code on Github\footnote{\url{https://github.com/diehlpkteaching/ParallelComputationMath/blob/master/code/13/Stencil4.ipynb}}\textsuperscript{,}\footnote{\url{https://github.com/STEllAR-GROUP/hpx/blob/master/examples/1d_stencil/1d_stencil_4.cpp}} and there is no need to add the HPX features by your own. Look at the following HPX features
\begin{enumerate}
\item \lstinline|std::unique_ptr<double[]> data_;|
\item \begin{lstlisting}
partition_data(std::size_t size)
      : data_(new double[size]), size_(size)
    {}
\end{lstlisting} \\
\item \textcolor{white}{test}

\begin{lstlisting}
// The partitioned operator, it invokes the heat operator above on all
// elements of a partition.
static partition_data heat_part(partition_data const& left,
        partition_data const& middle, partition_data const& right)
\end{lstlisting}

\item \lstinline|hpx::lcos::local::sliding_semaphore sem(nd);|

\item \begin{lstlisting}
next[i] = dataflow(
    hpx::launch::async, Op,
    current[idx(i, -1, np)], current[i], current[idx(i, +1, np)]
);
\end{lstlisting}

\item \begin{lstlisting}
// every nd time steps, attach additional continuation which will
// trigger the semaphore once computation has reached this point
if ((t % nd) == 0)
{
    next[0].then(
           [&sem, t](partition &&)
           {
               // inform semaphore about new lower limit
               sem.signal(t);
               });
            }

// suspend if the tree has become too deep, the continuation above
// will resume this thread once the computation has caught up
sem.wait(t);
\end{lstlisting}
\end{enumerate}

\end{enumerate}
and add a comment to the code why we need these features.

% ========== Continue adding items as needed
\doclicenseThis 

\end{document}
\grid
\grid
